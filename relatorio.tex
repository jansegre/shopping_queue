\documentclass[
  % -- opções da classe memoir --
  12pt,				% tamanho da fonte
  %openright,			% capítulos começam em pág ímpar (insere página vazia caso preciso)
  %twoside,			% para impressão em verso e anverso. Oposto a oneside
  oneside,			% para impressão em verso e anverso. Oposto a oneside
  a4paper,			% tamanho do papel. 
  % -- opções da classe abntex2 --
  %chapter=TITLE,		% títulos de capítulos convertidos em letras maiúsculas
  %section=TITLE,		% títulos de seções convertidos em letras maiúsculas
  %subsection=TITLE,	% títulos de subseções convertidos em letras maiúsculas
  %subsubsection=TITLE,% títulos de subsubseções convertidos em letras maiúsculas
  % -- opções do pacote babel --
  english,			% idioma adicional para hifenização
  brazil,				% o último idioma é o principal do documento
  % -- opções da classe ime-abntex2 --
  %brasao,
]{ime-abntex2}


% ---
% PACOTES
% ---

\usepackage[utf8]{inputenc}		% Codificacao do documento (conversão automática dos acentos)
\usepackage{float}                    % For improved placement of figures
\usepackage{placeins}                 % Keeps floats 'in their place' with \FloatBarrier
\usepackage{cmap}                     % Mapear caracteres especiais no PDF
\usepackage{lmodern}                  % Usa a fonte Latin Modern
\usepackage{amsmath}                  % Para usar blocos como "gather" de escrever matemática
%\usepackage{amsfonts}                 % Para usar fontes ams
%\usepackage{times}                    % Usa a fonte Times
\usepackage[T1]{fontenc}              % Selecao de codigos de fonte.
%\usepackage{lastpage}                 % Usado pela Ficha catalográfica
\usepackage{indentfirst}              % Indenta o primeiro parágrafo de cada seção.
\usepackage{color}                    % Controle das cores
\usepackage[table]{xcolor}            % Controle das cores em uma tabela
\usepackage{graphicx}                 % Inclusão de gráficos
\usepackage[portuguese]{algorithm2e}  % Para escrever algorítimos
\usepackage{lscape}                   % Para usar landscape
\usepackage{amsthm}                   % Para usar Definições e lemas
\usepackage{amssymb}                  % \mathbb{•}
\usepackage{caption}                  % dependencia de subcaption
%\usepackage{subcaption}               % Para usar \begin{subfigure} e colocar figuras (a) e (b) lado a lado
\usepackage{listings}                 % Para usar \lstinputlisting e incluir código
\newtheorem{teo}{Teorema}[section]    % Renomear o comando
\newtheorem{defi}[teo]{Definição}
% o seguinte é necessário para corrigir um bug no algorithm2e
% http://tex.stackexchange.com/questions/113325/problem-with-algorithm2e-and-portuguese-option
\SetKwFor{Para}{para}{fa\c{c}a}{fimpara}
\SetKwBlock{Inicio}{início}{fim}
% não era pra precisar do seguinte, mas precisa para traduzir a captions dos algoritmos
\renewcommand{\algorithmcfname}{Algoritmo}%
% ---

% ---
% Pacotes de citacoes
% ---
% O seguinte serve para mostrar as referencias reversas na bibliografia
% \usepackage[brazilian,hyperpageref]{backref}	 % Paginas com as citações na bibl
\usepackage[alf]{abntex2cite}	% Citações padrão ABNT

% --- 
% CONFIGURAÇÕES DE PACOTES
% --- 

% ----
% Configuração LISTING
% ----
\def\lstlistingname{Arquivo}
\lstset{                      % general command to set parameter(s)
  basicstyle=\footnotesize,   % print whole listing small. Possible values:
                              %   \footnotesize, \small, \itshape, \ttfamily
  keywordstyle=\color{black}\bfseries\underbar,
                              % underlined bold black keywords
  identifierstyle=,           % nothing happens
  commentstyle=\color{white}, % white comments
  stringstyle=\ttfamily,      % typewriter type for strings
  showstringspaces=false      % no special string spaces
}


% ---
% Configurações do pacote backref
% Usado sem a opção hyperpageref de backref
%\renewcommand{\backrefpagesname}{Citado na(s) página(s):~}
% Texto padrão antes do número das páginas
%\renewcommand{\backref}{}
% Define os textos da citação
%\renewcommand*{\backrefalt}[4]{
%	\ifcase #1 %
%		Nenhuma citação no texto.%
%	\or
%		Citado na página #2.%
%	\else
%		Citado #1 vezes nas páginas #2.%
%	\fi}%
% ---

% as imagens ficam nesse diretório
\graphicspath{{img/}}


% ---
% Informações de dados para CAPA e FOLHA DE ROSTO
% ---
\titulo{Determinação do Número Ótimo de Caixas de Um Supermercado}
\autor{Jan Segre\\Victor Bramigk}
\local{Rio de Janeiro}
\data{Outubro de 2015}
\orientador{TC Salles}{Ph.D., do IME}
\instituicao{%
  Instituto Militar de Engenharia
  \par
  Seção de Computação
  \par
  Graduação em Engenharia de Computação
}
\tipotrabalho{Simulação}
% O preambulo deve conter o tipo do trabalho, o objetivo,
% o nome da instituição e a área de concentração 
\preambulo{Trabalho da Disciplina de Simulação e Análise de Desempenho do Curso
  de Graduação em Engenharia de Computação do Instituto Militar de Engenharia.}
% ---


% ---
% Configurações de aparência do PDF final

% alterando o aspecto da cor azul
%\definecolor{blue}{RGB}{41,5,195}
\definecolor{blue}{RGB}{0,0,0}

% informações do PDF
\makeatletter
\hypersetup{%
  %pagebackref=true,
  pdftitle={\@title},
  pdfauthor={\@author},
  pdfsubject={\imprimirpreambulo},
  pdfcreator={LaTeX with abnTeX2},
  pdfkeywords={ime}{robocup}{analise}{logs}{rede neural},
  colorlinks=true,		% false: boxed links; true: colored links
  linkcolor=blue,			% color of internal links
  citecolor=blue,			% color of links to bibliography
  filecolor=magenta,		% color of file links
  urlcolor=blue,
  bookmarksdepth=4
}
\makeatother
% ---

% ---
% Espaçamentos entre linhas e parágrafos 
% ---

% O tamanho do parágrafo é dado por:
%\setlength{\parindent}{1.3cm}

% Controle do espaçamento entre um parágrafo e outro:
%\setlength{\parskip}{0.2cm}  % tente também \onelineskip
\setlength{\parskip}{\onelineskip}

% ---
% compila o indice
% ---
\makeindex
% ---

% ----
% Início do documento
% ----
\begin{document}

% Retira espaço extra obsoleto entre as frases.
%\frenchspacing

% ----------------------------------------------------------
% ELEMENTOS PRÉ-TEXTUAIS
% ----------------------------------------------------------
% \pretextual

% ---
% Capa
% ---
\imprimircapa
%\input{pre_textuais/capa}
% ---

% ---
% Folha de rosto
% (o * indica que haverá a ficha bibliográfica)
% ---
%\imprimirfolhaderosto*
%\input{pre_textuais/folha_de_rosto}
% ---

% ---
% Inserir a ficha bibliografica
% ---

% Isto é um exemplo de Ficha Catalográfica, ou ``Dados internacionais de
% catalogação-na-publicação''. Você pode utilizar este modelo como referência. 
% Porém, provavelmente a biblioteca da sua universidade lhe fornecerá um PDF
% com a ficha catalográfica definitiva após a defesa do trabalho. Quando estiver
% com o documento, salve-o como PDF no diretório do seu projeto e substitua todo
% o conteúdo de implementação deste arquivo pelo comando abaixo:
%
% \begin{fichacatalografica}
%     \includepdf{fig_ficha_catalografica.pdf}
% \end{fichacatalografica}
%\imprimirfichacatalografica
%{629.892}
%{S455h}
%{Segre, Jan}
%{Projeto de Fim de Curso (PFC)}
%{%
%  1. Curso de engenharia da computação - Projeto Final de Curso.
%  2. Robótica.
%  3. Minimax.
%  I. Bramigk, Victor.
%  II. Rosa, Paulo Fernando Ferreira.
%  III. Bruno Eduardo Madeira.
%  IV. \@title.
%  V. Instituto Militar de Engenharia.
%}
%{Técnicas de IA aplicadas a sistemas multiagentes cooperativos e
%competitivos}
% ---

% ---
% Inserir errata
% ---
%\begin{errata}
%Elemento opcional da \citeonline[4.2.1.2]{NBR14724:2011}. Exemplo:
%
%\vspace{\onelineskip}
%
%FERRIGNO, C. R. A. \textbf{Tratamento de neoplasias ósseas apendiculares com
%reimplantação de enxerto ósseo autólogo autoclavado associado ao plasma
%rico em plaquetas}: estudo crítico na cirurgia de preservação de membro em
%cães. 2011. 128 f. Tese (Livre-Docência) - Faculdade de Medicina Veterinária e
%Zootecnia, Universidade de São Paulo, São Paulo, 2011.
%
%\begin{table}[htb]
%\center
%\footnotesize
%\begin{tabular}{|p{1.4cm}|p{1cm}|p{3cm}|p{3cm}|}
%  \hline
%   \textbf{Folha} & \textbf{Linha}  & \textbf{Onde se lê}  & \textbf{Leia-se}  \\
%    \hline
%    1 & 10 & auto-conclavo & autoconclavo\\
%   \hline
%\end{tabular}
%\end{table}
%
%\end{errata}
% ---

% ---
% Inserir folha de aprovação
% ---
%
% Isto é um exemplo de Folha de aprovação, elemento obrigatório da NBR
% 14724/2011 (seção 4.2.1.3). Você pode utilizar este modelo até a aprovação
% do trabalho. Após isso, substitua todo o conteúdo deste arquivo por uma
% imagem da página assinada pela banca com o comando abaixo:
%
% \includepdf{folhadeaprovacao_final.pdf}
%
%\convidadoum{Bruno Eduardo Madeira}{M.Sc., do IME}
%\convidadodois{Julio Cesar Duarte}{D.Sc., do IME}
%\imprimirfolhadeaprovacao{18 de novembro de 2014}
% ---

% ---
% Dedicatória
% ---
%\begin{dedicatoria}
%   \vspace*{\fill}
%   \centering
%   \noindent
%   \textit{ Este trabalho é dedicado às crianças adultas que,\\
%   quando pequenas, sonharam em se tornar cientistas.} \vspace*{\fill}
%\end{dedicatoria}
% ---

% ---
% Agradecimentos
% ---
%\begin{agradecimentos}
%Os agradecimentos principais são direcionados à Gerald Weber, Miguel Frasson,
%Leslie H. Watter, Bruno Parente Lima, Flávio de Vasconcellos Corrêa, Otavio Real
%Salvador, Renato Machnievscz\footnote{Os nomes dos integrantes do primeiro
%projeto abn\TeX\ foram extraídos de
%\url{http://codigolivre.org.br/projects/abntex/}} e todos aqueles que
%contribuíram para que a produção de trabalhos acadêmicos conforme
%as normas ABNT com \LaTeX\ fosse possível.
%
%Agradecimentos especiais são direcionados ao Centro de Pesquisa em Arquitetura
%da Informação\footnote{\url{http://www.cpai.unb.br/}} da Universidade de
%Brasília (CPAI), ao grupo de usuários
%\emph{latex-br}\footnote{\url{http://groups.google.com/group/latex-br}} e aos
%novos voluntários do grupo
%\emph{\abnTeX}\footnote{\url{http://groups.google.com/group/abntex2} e
%\url{http://abntex2.googlecode.com/}}~que contribuíram e que ainda
%contribuirão para a evolução do \abnTeX.
%
%\end{agradecimentos}
% ---

% ---
% Epígrafe
% ---
%\begin{epigrafe}
%    \vspace*{\fill}
%	\begin{flushright}
%		\textit{``Não vos amoldeis às estruturas deste mundo, \\
%		mas transformai-vos pela renovação da mente, \\
%		a fim de distinguir qual é a vontade de Deus: \\
%		o que é bom, o que Lhe é agradável, o que é perfeito.\\
%		(Bíblia Sagrada, Romanos 12, 2)}
%	\end{flushright}
%\end{epigrafe}
% ---

% ---
% RESUMOS
% ---
%\input{doc/resumo}

% ---
% inserir o sumario
% ---
\pdfbookmark[0]{\contentsname}{toc}
\tableofcontents*
\cleardoublepage
% ---

% ---
% inserir lista de ilustrações
% ---
%\pdfbookmark[0]{\listfigurename}{lof}
%\listoffigures*
%\cleardoublepage
% ---

% ---
% inserir lista de tabelas
% ---
\pdfbookmark[0]{\listtablename}{lot}
\listoftables*
\cleardoublepage
% ---

% ---
% inserir lista de abreviaturas e siglas
% ---
%\begin{siglas}
%  \item[IA] Inteligência Artificial
%  \item[SSL] \textit{Small Size League}
%  \item[Robocup] \textit{Robotics World Cup}
%  \item[RoboIME] Equipe de alunos do Laboratório de Robótica do IME
%\end{siglas}
% ---

% ---
% inserir lista de símbolos
% ---
%\begin{simbolos}
%  \item[$ \Gamma $] Letra grega Gama
%  \item[$ \Lambda $] Lambda
%  \item[$ \zeta $] Letra grega minúscula zeta
%  \item[$ \in $] Pertence
%\end{simbolos}
% ---


% ----------------------------------------------------------
% ELEMENTOS TEXTUAIS
% ----------------------------------------------------------
\textual

% ----------------------------------------------------------
% Introdução
% ----------------------------------------------------------
\chapter{Introdução}
% área, explicação, relevância, referências, 
% objetivo e aplicação do problema
Frequêntemente são vistos lojas com filas grandes. Então, surge
a pergunta: quantos caixas são necessários para que a fila não
seja superior a um determinado valor?

Essa pergunta é de fundamental importância para os donos desses
comércios, já que se o número não for o suficiente, lucro será
perdido devido a evasão de clientes. Também, devido ao custo
de se contratar funcionários, não é interessante que o número
de caixas seja maior do que o necessário, de modo a se evitar
um gasto desnecessário com pessoal. Pode-se também analisár
quais os horários e datas mais críticos para que sejam alocados
funcionários quando necessário.

Este trabalho tratará do caso mais simples do problema descrito
anteriormente utilizando-se de uma simulação para encontrar
o número de caixas necessário para atingir-se o tamanho máximo
de fila desejado.

% vim: tw=80 et ts=2 sw=2 sts=2


% FIXME: FALTOU A SEÇÃO DE REVISÃO DE LITERATURA???

% Desenvolvimento
% ---------------
% intro

\chapter{Formulação do problema}
O objetivo deste trabalho é encontrar o número ótimo
de caixas que permitam que um supermercado não tenham filas
com tamanho maior que 5 pessoas. São consideradas duas estruturas
para resolução do problema. Na primeira as filas dos caixas são
independentes, e os clientes entram no caixa de menor fila.
Na segunda, os caixas possuem uma fila comum, assim como em um
caixa rápido.

Para cada um dos casos são apresentados a fila média máxima em cada
caso, o tempo médio de permanência no supermercado, o número mínimo
de caixas para atingir o objetivo desejado.

Restrições do problema:
\begin{itemize}
  \item O supermercado funciona 8 horas por dia e tem $n>2$ caixas
  \item Chegadas dos clientes é Poisson de média 10 por minuto
  \item Tempo de atendimento no caixa é Uniforme entre 2 e 6 minutos
  \item Tempo de compra de cada cliente é Uniforme entre 30 e 90 minutos
\end{itemize}

\chapter{Resultados}\label{cap:resul}
O problema descrito anteriormente foi simulado em \textit{python}. Os resultados
da simulação são apresentados na tabela \ref{tab:resul_5}.

% fila média
% tempo médio de permanência no supermercado,
% e máxima em cada caixa,
% número mínimo de caixas tal que fila < 5 pessoas
% intervalo de confiânça
% 
% Isso deve ser apresentado para:
%   - no. caixas < n*
%   - no. caixas = n*
%   - no. caixas > n*
%   onde n* é o número ótimo de caixas
\begin{table}[ht]
  \resizebox{\textwidth}{!}{%
    \begin{tabular}{|c|c|c|c|c|}
      \hline
      \rowcolor[gray]{.8}
      Nº caixa & $l_{med}$ fila & $t_{med}$ supermercado & $t_{med}$ fila & $t_{max}$ fila \\
      \hline
      % data
         1     &              &                      &              &              \\ \hline
         2     &              &                      &              &              \\ \hline
         3     &              &                      &              &              \\ \hline
         4     &              &                      &              &              \\ \hline
         5     &              &                      &              &              \\ \hline
    \end{tabular}
  }
  \caption{Resultado da simulação para 5 caixas.\label{tab:resul_5}}
\end{table}


% vim: tw=80 et ts=2 sw=2 sts=2

\chapter{Conclusão}
Conforme apresentado no capítulo \ref{cap:resul}, o número
de caixas altera drasticamente o tempo de atendimento.
O número mínimo de caixas para se ter uma fila máxima de 5
pessoas é 16.

Fica evidente o granho de produtividade no sistema,
evidenciando a importânica do estudo apresentado para
garantir o ganho máximo de produtividade do sistema.

% vim: tw=80 et ts=2 sw=2 sts=2


% ---
% Finaliza a parte no bookmark do PDF, para que se inicie o bookmark na raiz
% ---
\bookmarksetup{startatroot}%
% ---

%\addcontentsline{toc}{chapter}{Conclusão}
%\chapter{Conclusão}
Conforme apresentado no capítulo \ref{cap:resul}, o número
de caixas altera drasticamente o tempo de atendimento.
O número mínimo de caixas para se ter uma fila máxima de 5
pessoas é 16.

Fica evidente o granho de produtividade no sistema,
evidenciando a importânica do estudo apresentado para
garantir o ganho máximo de produtividade do sistema.

% vim: tw=80 et ts=2 sw=2 sts=2


% ----------------------------------------------------------
% ELEMENTOS PÓS-TEXTUAIS
% ----------------------------------------------------------
\postextual


% ----------------------------------------------------------
% Referências bibliográficas
% ----------------------------------------------------------
%\bibliographystyle{plainnat}%abbrvnat, unsrtnat, apsrev, rmpaps, IEEEtranN, achemso, rsc
%\bibliography{referencias}

% ----------------------------------------------------------
% Glossário
% ----------------------------------------------------------
%
% Consulte o manual da classe abntex2 para orientações sobre o glossário.
%
%\glossary

% ----------------------------------------------------------
% Apêndices
% ----------------------------------------------------------

% ---
% Inicia os apêndices
% ---
%\begin{apendicesenv}

% Imprime uma página indicando o início dos apêndices
%\partapendices

% ----------------------------------------------------------
%\chapter{Quisque libero justo}
% ----------------------------------------------------------
%\end{apendicesenv}
% ---


% ----------------------------------------------------------
% Anexos
% ----------------------------------------------------------

% ---
% Inicia os anexos
% ---
%\begin{anexosenv}

% Imprime uma página indicando o início dos anexos
%\partanexos

% ---
%\chapter{Morbi ultrices rutrum lorem.}
% ---

%\end{anexosenv}

%---------------------------------------------------------------------
% INDICE REMISSIVO
%---------------------------------------------------------------------

\printindex

\end{document}
% vim: tw=80 et ts=2 sw=2 sts=2
