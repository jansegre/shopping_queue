\chapter{Introdução}
% área, explicação, relevância, referências, 
% objetivo e aplicação do problema
Frequêntemente são vistos lojas com filas grandes. Então, surge
a pergunta: quantos caixas são necessários para que a fila não
seja superior a um determinado valor?

Essa pergunta é de fundamental importância para os donos desses
comércios, já que se o número não for o suficiente, lucro será
perdido devido a evasão de clientes. Também, devido ao custo
de se contratar funcionários, não é interessante que o número
de caixas seja maior do que o necessário, de modo a se evitar
um gasto desnecessário com pessoal. Pode-se também analisár
quais os horários e datas mais críticos para que sejam alocados
funcionários quando necessário.

Este trabalho tratará do caso mais simples do problema descrito
anteriormente utilizando-se de uma simulação para encontrar
o número de caixas necessário para atingir-se o tamanho máximo
de fila desejado.

% vim: tw=80 et ts=2 sw=2 sts=2
