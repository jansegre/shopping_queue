\chapter{Formulação do problema}
O objetivo deste trabalho é encontrar o número ótimo
de caixas que permitam que um supermercado não tenham filas
com tamanho maior que 5 pessoas. São consideradas duas estruturas
para resolução do problema. Na primeira as filas dos caixas são
independentes, e os clientes entram no caixa de menor fila.
Na segunda, os caixas possuem uma fila comum, assim como em um
caixa rápido.

Para cada um dos casos são apresentados a fila média máxima em cada
caso, o tempo médio de permanência no supermercado, o número mínimo
de caixas para atingir o objetivo desejado.

Restrições do problema:
\begin{itemize}
  \item O supermercado funciona 8 horas por dia e tem $n>2$ caixas
  \item Chegadas dos clientes é Poisson de média 10 por minuto
  \item Tempo de atendimento no caixa é Uniforme entre 2 e 6 minutos
  \item Tempo de compra de cada cliente é Uniforme entre 30 e 90 minutos
\end{itemize}

\chapter{Resultados}\label{cap:resul}
O problema descrito anteriormente foi simulado em \textit{python}. Os resultados
da simulação são apresentados na tabela \ref{tab:resul_5}.

% fila média
% tempo médio de permanência no supermercado,
% e máxima em cada caixa,
% número mínimo de caixas tal que fila < 5 pessoas
% intervalo de confiânça
% 
% Isso deve ser apresentado para:
%   - no. caixas < n*
%   - no. caixas = n*
%   - no. caixas > n*
%   onde n* é o número ótimo de caixas
\begin{table}[ht]
  \resizebox{\textwidth}{!}{%
    \begin{tabular}{|c|c|c|c|c|}
      \hline
      \rowcolor[gray]{.8}
      Nº caixa & $l_{med}$ fila & $t_{med}$ supermercado & $t_{med}$ fila & $t_{max}$ fila \\
      \hline
      % data
         1     &              &                      &              &              \\ \hline
         2     &              &                      &              &              \\ \hline
         3     &              &                      &              &              \\ \hline
         4     &              &                      &              &              \\ \hline
         5     &              &                      &              &              \\ \hline
    \end{tabular}
  }
  \caption{Resultado da simulação para 5 caixas.\label{tab:resul_5}}
\end{table}


% vim: tw=80 et ts=2 sw=2 sts=2
